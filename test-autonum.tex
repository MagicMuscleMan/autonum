\documentclass{article}

% \usepackage[english,ngerman]{babel}
% \usepackage[utf8]{inputenx}
% \usepackage[utf8x]{inputenx}

\usepackage{amsmath}
\usepackage[hypertexnames=false]{hyperref}
\usepackage{cleveref}
\usepackage{autonum}

\newcounter{Hequation}
\renewcommand{\theHequation}{\thechapter.\arabic{Hequation}}
\makeatletter
\g@addto@macro\equation{\stepcounter{Hequation}}

\listfiles

\@ifpackageloaded{autonum}{%
	\def\ifautonum#1{#1}%
}{%
	\def\ifautonum#1{}%
}

\@ifpackageloaded{cleveref}{%
	\def\ifcleveref#1{#1}%
	\crefname{inequation}{ineq.}{ineqs.}
}{%
	\def\ifcleveref#1{}%
}

\makeatother

\begin{document}

% \tracingall
% \makeatletter
% \show\equation
% \show\autonum@equationOld
% \show\mathdisplay
% \show\mathdisplay@pop

% \begin{equation}\label{dd}a\end{equation}%\ref{dd}
% % \stop
% \begin{equation}\label{ee}b\end{equation}
	\section*{Tests}
	\begin{itemize}
		\item Additionally, test that deactivating the package does not result in compile errors during the next run if only basic features are used.
		\item Additionally, test that everything works with and without the above inputenx package (after deactivating the very strange label below).
		\item Having a referenced equation with reference before \ref{referenceBefore}
			\begin{equation}\label{referenceBefore}
				d - d = 0
			\end{equation}
			\begin{equation}\label{referenceNo}
				d - d = 0
			\end{equation}
		\ref{b}\begin{equation}\label{a}a\end{equation}\begin{equation}\label{b}b\end{equation}
		\item Having a referenced equation with reference after
			\begin{equation}\label{referenceAfter}
				c^2 = c c
			\end{equation}
			\ref{referenceAfter}
		\item Having an unlabeled equation
			\begin{equation}\label{abc}
				a^2 + b^2 = c^2
			\end{equation}
		\item Having a labeled, but unreferenced equation
			\begin{equation}\label{unreferenced}
				\sqrt{a}
			\end{equation}
		\item Having a labeled equation with a very strange label \ref{äöüÄÖÜß?:, 3075µ!/§} does only work without package inputenx
			\begin{equation}\label{äöüÄÖÜß?:, 3075µ!/§}
				\sqrt{b}
			\end{equation}
		\item Having a labeled equation with a colon in the label \ref{label:colon}
			\begin{equation}\label{label:colon}
				\sqrt{c}
			\end{equation}
		\item Having an equation with a following label with a colon in the label \ref{labelAfter:colon}
			\begin{equation}
				\sqrt{d}\label{labelAfter:colon}
			\end{equation}
		\item Having an equation with a following label with a colon in the label
			\begin{equation}
				\sqrt{e}\label{referenceAfter:colon}
			\end{equation}
			and referencing \ref{referenceAfter:colon} only afterwards
		\item Having a labeled equation with umlauts in the label \ref{äöüÄÖÜßLabel}
			\begin{equation}\label{äöüÄÖÜßLabel}
				\sqrt{c}
			\end{equation}
		\item Check for spurious whitespace around reference (\ref{checkWhitespace})
			\begin{equation}\label{checkWhitespace}
				b_c
			\end{equation}
		\item Check if the starred version of ref does also work (\ref*{checkStarred})
			\begin{equation}\label{checkStarred}
				c_D
			\end{equation}
		\ifcleveref{
			\item Check if the starred version of cref does also work (\cref*{checkStarredCref})
				\begin{equation}\label{checkStarredCref}
					d_E
				\end{equation}
		}
		\item Placing the number in long equations \ref{long}
			\begin{equation}\label{long}
				\sum\sum\sum\sum\sum\sum\sum\sum\sum\sum\sum\sum\sum\sum\sum\sum\sum\sum\sum a
			\end{equation}
		\ifautonum{
			\item Printing the number without referencing (needs autonum)
				\begin{equation+}
					E = mgh
				\end{equation+}
		}
		\item Using a ref inside a caption
			\begin{figure}
				ref
				\caption{\ref{long}}
			\end{figure}
		\ifcleveref{
			\item Using a cref inside a caption
				\begin{figure}
					cref
					\caption{\cref{long}}
				\end{figure}
			\item Using cref with one argument
				\begin{equation}\label{crefOne}
					g
				\end{equation}
				\cref{crefOne}
			\item Using cref with two arguments
				\begin{equation}\label{crefTwo}
					cr = ef
				\end{equation}
				\cref{crefOne,crefTwo}
			\ifautonum{
				\item Using otherwise unused cref with two arguments (needs autonum)
					\[\label{crefThree}
						cr = ef
					\]
					\[\label{crefFour}
						cr = ef
					\]
					\cref{crefThree,crefFour}
			}
			\item Using cref with a custom type \cref{myInequation} and thus an optional argument in the label command
				\begin{equation}\label[inequation]{myInequation}
					a < b
				\end{equation}
			\item Using an unused cref with a custom type and thus an optional argument in the label command
				\begin{equation}\label[inequation]{myUnusedInequation}
					d < c
				\end{equation}
		}
		\item Using align \ref{alignOne}, \ref{alignThree}
			\begin{align}
				a\label{alignOne}\\
				b\label{alignTwo}\\
				c\label{alignThree}
			\end{align}
		\item Using gather \ref{gatherOne}, \ref{gatherThree}
			\begin{gather}
				a\label{gatherOne}\\
				b\label{gatherTwo}\\
				c\label{gatherThree}
			\end{gather}
		\item Using multline without referencing
			\begin{multline}
				a\\
				c\label{multlineUnreferenced}
			\end{multline}
		\item Using multline with referencing \ref{multlineReferenced}
			\begin{multline}
				a\\
				c\label{multlineReferenced}
			\end{multline}
		\item Using flalign with referencing \ref{flalignReferenced}
			\begin{flalign}
				a\\
				c\label{flalignReferenced}
			\end{flalign}
		\item Using alignat with referencing \ref{alignatReferenced}
			\begin{alignat}{4}
				x &= yy & \implies & y &= x \label{alignatUnreferenced}\\
				y &= z & \implies & z &= y \label{alignatReferenced}
			\end{alignat}
		\item short one-line shortcut \[n\]
% 		\item shortcut environment with two lines, referencing \ref{firstShortcut} \[n_1\label{firstShortcut} \\ n_2\]
		\ifautonum{
			\item align, numbering always \begin{align+} a=l \end{align+} (needs autonum)
			\item gather, numbering always \begin{gather+} g=a \end{gather+} (needs autonum)
			\item multline, numbering always (and avoiding overfull hbox warning) \begin{multline+} m=u\line(1,0){220}=v \end{multline+} (needs autonum)
			\item equation, numbering always \begin{equation+} e=q \end{equation+} (needs autonum)
		}
		\item shortcut and split \ref{split} \[ \label{split}\begin{split} s \\ p \end{split} \] (needs autonum)
		\item equation and split \ref{splitEquation} \begin{equation} \label{splitEquation}\begin{split} s \\ p \end{split} \end{equation}
% 		\item super-short \[\(a+b\\d+e\)\]
% 		\item Using aligned \ref{alignedOne}, \ref{alignedThree}
% 			\begin{equation}
% 				\begin{aligned}
% 					a\label{alignedOne}\\
% 					b\label{alignedTwo}\\
% 					c\label{alignedThree}
% 				\end{aligned}
% 			\end{equation}
% 		\item Using gathered \ref{gatheredOne}, \ref{gatheredThree}
% 			\begin{equation}
% 				\begin{gathered}
% 					a\label{gatheredOne}\\
% 					b\label{gatheredTwo}\\
% 					c\label{gatheredThree}
% 				\end{gathered}
% 			\end{equation}
	\end{itemize}
	\section{Using ref in section \ref{i1}}\label{i1} text
	\ifcleveref{
		\section{Using cref in \cref{i2}}\label{i2} text
		\begin{figure}
			\caption{Ref 2: \cref{i2} and \ref{i2}}
		\end{figure}
	}
\tableofcontents
\listoffigures
\end{document}