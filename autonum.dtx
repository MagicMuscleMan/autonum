% \iffalse meta-comment
% !TEX program  = pdfLaTeX
%<*internal>
\iffalse
%</internal>
%<*readme>
----------------------------------------------------------------
autonum --- Automatic number referenced equations
E-mail: pat_h@web.de
Released under the LaTeX Project Public License v1.3c or later
See http://www.latex-project.org/lppl.txt
----------------------------------------------------------------

This is the autonum package, automatically numbering only the equations which
are referenced.
This package is similar to mathtools' showonlyrefs option, but without
introducing the bug to overwrite long equations with the equation number.

Usage: \usepackage{autonum}

Everything else should happen automatically.
To use autonum together with other packages, load autonum last.
%</readme>
%<readme>\endbatchfile
%<*internal>
\fi
\def\nameofplainTeX{plain}
\ifx\fmtname\nameofplainTeX\else
  \expandafter\begingroup
\fi
%</internal>
%<*install>
\input docstrip.tex
\keepsilent
\askforoverwritefalse
\preamble
----------------------------------------------------------------
autonum --- Automatic number referenced equations
E-mail: pat_h@web.de
Released under the LaTeX Project Public License v1.3c or later
See http://www.latex-project.org/lppl.txt
----------------------------------------------------------------

\endpreamble
\postamble

Copyright (C) 2012 by Patrick Häcker <pat_h@web.de>

This work may be distributed and/or modified under the
conditions of the LaTeX Project Public License (LPPL), either
version 1.3c of this license or (at your option) any later
version.  The latest version of this license is in the file:

http://www.latex-project.org/lppl.txt

This work is "maintained" (as per LPPL maintenance status) by
Patrick Häcker.

This work consists of the file  autonum.dtx
and the derived files           autonum.ins,
                                autonum.pdf and
                                autonum.sty.

\endpostamble
\usedir{tex/latex/autonum}
\generate{
  \file{autonum.sty}{\from{autonum.dtx}{package}}
}
%</install>
%<install>\endbatchfile
%<*internal>
\usedir{source/latex/autonum}
\generate{
  \file{autonum.ins}{\from{autonum.dtx}{install}}
}
\nopreamble\nopostamble
\usedir{doc/latex/autonum}
\generate{
  \file{README.txt}{\from{autonum.dtx}{readme}}
}
\ifx\fmtname\nameofplainTeX
  \expandafter\endbatchfile
\else
  \expandafter\endgroup
\fi
%</internal>
%<*package>
\NeedsTeXFormat{LaTeX2e}
\ProvidesPackage{autonum}[2012/08/25 v0.3.1 autonum LaTeX package]
\PassOptionsToPackage{hypertexnames=false}{hyperref}
\RequirePackage{etoolbox}
\RequirePackage{amsmath}
\RequirePackage[absolute]{textpos}
%</package>
%<*driver>
\documentclass{ltxdoc}
\usepackage[utf8]{inputenx}
\usepackage[T1]{fontenc}
\usepackage{lmodern}
\usepackage{amsmath}
\usepackage{\jobname}
\usepackage[numbered]{hypdoc}
\usepackage{autonum}%
\hypersetup{pdftitle=The autonum package}
\EnableCrossrefs
\CodelineIndex
\RecordChanges
\begin{document}
  \DocInput{\jobname.dtx}
\end{document}
%</driver>
% \fi
%
%\GetFileInfo{\jobname.sty}
%
%\title{^^A
%  \textsf{autonum} --- automatic number referenced equations\thanks{^^A
%    This file describes version \fileversion, last revised \filedate.^^A
%  }^^A
%}
%\author{^^A
%  Patrick Häcker\thanks{E-mail: pat\_h@web.de}^^A
%}
%\date{Released \filedate}
%
%\maketitle
%
%\changes{v0.1}{2012/05/26}{First public release}
%\changes{v0.2}{2012/06/21}{Support multi-line environments. Redefine \cmd{\[} to use the new \cmd{\equation}.}
%\changes{v0.3}{2012/08/16}{Support alignat and flalign environments. Fix handling of special characters in label names. Fix handling of multiple label names in \cmd{\cref} command.}
%\changes{v0.3.1}{2012/08/25}{Fix of leading whitespace in references. Fix of underfull hbox.}
%
% \section{Introduction}
% With \LaTeX, the user has to decide manually to not number an equation by adding a star in the math environment. Authors who do not want to think about the numbering during the writing (and maybe they should not), often use the environments without stars. As default, these users get all equations numbered, although there are \href{http://tex.stackexchange.com/q/29267/7323}{different opinions} about what should be numbered.
%
% The other automatic possibility one can think of is to number only the referenced equations. The \href{http://www.ctan.org/tex-archive/macros/latex/contrib/mh/}{mathtools} package's option \texttt{showonlyrefs} seems to be the solution for those who want to have the referenced equations numbered only. Unfortunately for \href{http://www.ctan.org/pkg/amsmath}{amsmath} users this also means a step backwards, as the numbering can overwrite parts of the equation according to its documentation. Generally, this options seems to be quite unreliable as it is incompatible with the \href{http://www.ctan.org/tex-archive/macros/latex/contrib/cleveref/}{cleveref} package. The autonum package implements the numbering of referenced equations only without this deficiencies.
%^^A This currently does not work with cleveref, as equations without a number increment the equation counter, too. As placing the labels does not always work correctly, \cmd{\mathtoolsset{showonlyrefs}} is not an option (see mathtools documentation, bottom of page 10).
%
% \section{Usage and Examples}
% To get the automatic numbering of referenced equations, simply load the package:
%
% \vspace{0.5em}
% \cmd{\usepackage\{autonum\}}
% \vspace{0.5em}
%
% The recommended style is to add a label to each logical equation. Then, simply use the references as needed. Never use the starred forms when defining an equation as they do not make sense with autonum and are deactivated. You can use the (redefined) \texttt{equation} or \cmd{\[}-\cmd{\]}-environment in most cases, if you add the appropriate \cmd{\begin\{split\}}-\cmd{\end\{split\}} pairs when needed. Sometimes, an \texttt{align}, \texttt{multline}, \texttt{gather}, \texttt{flalign} or \texttt{alignat} environment is needed.  Do \href{http://tug.org/pracjourn/2006-4/madsen/madsen.pdf}{not use \texttt{eqnarray}}.^^A, rarely use the \texttt{aligned} or \texttt{gathered} environments.
%
% Please keep in mind, that using autonum might not always be a good thing. If you hand-in a paper for review with many equations on a page, you might avoid using autonum or if you do, you probably should reference most of your equations or activate line numbers.
%
%\begin{macro}{equation}
% The following examples show the results of the equation environment after loading the package. Now let's reference the third equation: \ref{alreadyReferenced}.
% \begin{equation}
% 	1 = 1\quad(\text{equation without label})
% \end{equation}
%
% \begin{equation}\label{notReferenced}
% 	2 = 2\quad(\text{equation with label, not referenced})
% \end{equation}
%
% \begin{equation+}\label{alreadyReferenced}
% 	3 = 3\quad(\text{equation with label, already referenced})
% \end{equation+}
%
% \begin{equation+}\label{referencedLater}
% 	4 = 4\quad(\text{equation with label, referenced later})
% \end{equation+}
% Now let's reference the fourth equation: \ref{referencedLater}. The first and the second equations do not get a number, as they are not referenced.
%
% If you want to try an example yourself, you can take this one, where only the first equation gets a number:
%
% \vspace{0.5em}
% \noindent\cmd{\documentclass\{minimal\}}\\
% \cmd{\usepackage\{autonum\}}\\
% \cmd{\begin\{document\}}\\
% \indent\cmd{\begin\{equation\}}\cmd{\label\{a\}}\\
% \indent\indent a\\
% \indent\cmd{\end\{equation\}}\\
% \indent\cmd{\begin\{equation\}}\cmd{\label\{b\}}\\
% \indent\indent b\\
% \indent\cmd{\end\{equation\}}\\
% \indent\cmd{\ref\{a\}}\\
% \cmd{\end\{document\}}
% \vspace{0.5em}
%
% To get the correct references up to three compilation runs are necessary when the autonum package is used (as always in \TeX\ this cannot be guaranteed, so in rare cases even that is not enough). This is one additional run compared to the default referencing mode, as one run is needed to check if an equation is used. This is probably not avoidable, as the information which equation should have a number is simply not always available in the first run while layouting the equation. Thus, the equation's number can change in the second run. For the reference command, this information is only stable in the third run, as the equations influence each other.
%\end{macro}
%
% \begin{macro}{\[ et al.}
% Instead of equation you may also use one of the following environments: \texttt{align}, \texttt{gather}, \texttt{multline}, \texttt{flalign}. ^^A, \texttt{aligned}, \texttt{gathered}. ^^A The \texttt{multlined} environment is patched, if it is available.
% As the commands \cmd{\[} and \cmd{\]} are useless when using autonum, they are redefined to be an alias of \texttt{begin\{equation\}} and \texttt{end\{equation\}}, respectively. ^^A, instead either the \texttt{equation} or the \texttt{align} environments are used, depending on the number of lines in the equation.
% Without adding labels or referencing added labels, the new \cmd{\[}-\cmd{\]}-environment behaves like the old one. ^^A if the equation is only one line. The new definition inherits all the multi-line and alignment capabilities from align and thus is recommended as default while writing. If the environment's content has only one line, the equation environment is used, which has a \href{http://tex.stackexchange.com/questions/321/align-vs-equation}{smaller vertical space} for short lines above the environment compared to align.
% \end{macro}
%
%\begin{macro}{equation+}
% In the rare case, that an equation is very important and not references within the text, but some other person wants to reference to that equation, you can use \cmd{\begin\{equation+\}} and \cmd{\end\{equation+\}}. The equation is then numbered in all cases (having a label or not, being referenced or not). This feature has been added, so that all \href{http://tex.stackexchange.com/a/52287}{three referencing practices} are supported in \LaTeX\ with as much automation as possible. The other math environments support a \texttt{+}-variant, too.
%\end{macro}
%
%\begin{macro}{equation*}
% The starred versions have been deleted, to avoid accidental use. Use the versions without star and without adding a label, to avoid that an equation gets a number.
%\end{macro}
%
% \section{Compatibility}
% \subsection{Load Order}
% As other packages might break autonum, it should be loaded very late. Normally, amsmath is loaded automatically to make use of the more advanced math environments. Nevertheless, to use other packages, it might be necessary to load amsmath manually. To use autonum with amsmath, hyperref and cleveref for example, the order must be amsmath $\rightarrow$ hyperref $\rightarrow$ cleveref $\rightarrow$ autonum, as cleveref may not be loaded after amsmath. If a wrong order has been active, it might be necessary to delete the aux file manually to get rid of compile errors.  The support of autonum without amsmath has been discontinued due to missing demand.
%
% \subsection{Hyperref}
% The hyperref package must be loaded with the option \texttt{hypertexnames=false} to work when autonum is used.\\
% Problem description: Generally, the autonum package is careful not to break other packages, but with autonum there are non-starred mathematical environment which do not increase the equation counter. This leads to the following warning in hyperref when more than one equation is used: "destination with the same identifier (name{equation.0.1}) has been already used, duplicate ignored". Additionally, the link anchors do not work correctly (with a reference followed by a labeled equation, there is a warning in tracing mode, too. Hyperref seems to increase \LaTeX's equation's counter (called \texttt{equation}) only if a equation is referenced. This counter may not be modified to avoid problems with Hyperref, as this would change the displayed equation number (and it does not work anyway). This problem is independent of cleveref. The problem might be solvable by modifying \cmd{\begin\{equation\}} or \cmd{\end\{equation\}}. Using \texttt{gather} instead of \texttt{equation} removes the warning, but hyperlinks still do not work.
%
% \subsection{Reference commands}
% As default the \cmd{\ref} and cleveref's \cmd{\cref} commands are supported. Support for new reference commands can be added by executing\\
% \cmd{\let}\cmd{\ref}\cmd{\NewReferenceCommand}.\\
% If the original \cmd{\ref} command should not be overwritten, you have the following choices. Please mind that these interfaces might change in the future. Please also mind the \cmd{\makeatletter} and \cmd{\makeatother} command before and after the commands, respectively.
%
%\begin{macro}{\autonum@generatePatchedReference}
% For normal reference commands expecting a single label name as an argument, you can use \cmd{\autonum@generatePatchedReference}\{\cmd{\NewReferenceCommand}\}.
%\end{macro}
%
%\begin{macro}{\autonum@generatePatchedReferenceCSL}
% For reference commands expecting a comma separated list of label names as an argument, you can use\\
% \cmd{\autonum@generatePatchedReferenceCSL}\{\cmd{\NewReferenceCommand}\}.
%\end{macro}
%
%\begin{macro}{\autonum@generatePatchedReferenceGeneral}
% For reference commands expecting a general data structure containing label names as an argument, you can use\\
% \cmd{\autonum@generatePatchedReferenceGeneral}\{\cmd{\NewReferenceCommand}\}\\
% \{\cmd{\SplitAndLoopMacro}\}.\\
% The macro \cmd{\SplitAndLoopMacro} acts as a function expecting the to-be-called function as the first argument and the data structure as the second argument. It must call the function given in the first argument for each label name given in the data structure of the second argument. An example is the \cmd{\forcsvlist} macro from the etoolbox package, which is used to implement \cmd{\autonum@generatePatchedReferenceCSL}.
%\end{macro}
%
% \section{Further Ideas}
% \begin{itemize}
% \item A similar approach could be used for figure and table environments to print a warning, if such an environment is not referenced. With the subfig package, the solution would be harder, as a figure or table may be unreferenced, if all subfloats are referenced. Similarly, a subfloat may be unreferenced, if its parent environment is referenced. So the warning should only be printed if an unreferenced parent environment either does not have any child environment or if there is an unreferenced child environment.
% \item A warning could be printed, if another compile is necessary.
% \item A "*" could be used instead of a "+".
% \item If a reference is used before the label is defined, the reference information is saved in a variable and can be used later in the current run when processing the label. It does not have to be saved to the aux file. If a reference is used after defining the label, the reference information is saved in the aux file and can be used in the next run when processing the label. The information does not have to be saved into a variable, as a label must only be defined once and the definition has already happened. Unfortunately, this would make it necessary to distinguish between definitions in the current and in the last run, as otherwise this leads either to oscillation or to defining everything in the end (depending if csdefaux or csdefall is used in the not-defined case), which is probably not worth the effort.
% \item The command \cmd{\(} could be an alias for \texttt{begin\{split\}} and \cmd{\)} could be an alias for \texttt{end\{split\}}
% \item This \href{http://tex.stackexchange.com/a/5652}{trick} might be handy
% \item The \cmd{\crefrange} might be supported. As this would require a lot of work it will only be done if multiple people show interest and there are really convincing real-world examples where using \cmd{\crefrange} is superior to using \cmd{\cref}. Patches are, of course, welcome, too.
% \end{itemize}
%
% \section{Contributions}
% \begin{itemize}
%     \item David Carlisle
%         \begin{itemize}
%             \item explained that amsmath environments \href{http://tex.stackexchange.com/a/59088/7323}{are executed twice} what lead to the support of the amsmath environments
%         \end{itemize}
%     \item Fg Nu
%         \begin{itemize}
%             \item highlighted, that special characters in label names must be supported what lead to the support of special characters
%         \end{itemize}
%     \item Joseph Wright
%         \begin{itemize}
%             \item created a \href{http://tex.stackexchange.com/a/64271/7323}{correctly working} \cmd{\csxdefaux} what lead to the support of special characters
%         \end{itemize}
%     \item Jonas Nyrup
%         \begin{itemize}
%             \item reported a bug occuring when using \cmd{\cref} with comma separated arguments what lead to its fix
%             \item started an interesting discussion if cleveref's \cmd{\crefrange} command should be supported in autonum or not
%         \end{itemize}
%     \item Toby Cubitt
%         \begin{itemize}
%             \item highlighted, that there are some users who might find valid use of cleveref's \cmd{\crefrange} command
%         \end{itemize}
%     \item Marko Pinteric
%         \begin{itemize}
%             \item found an underfull hbox error what lead to its removal
%             \item found the existance of a spurious whitespace problem what lead to its removal
%         \end{itemize}
%     \item Heiko Oberdiek
%         \begin{itemize}
%             \item found the reason of spurious whitespaces what lead to their removal
%             \item found an underfull hbox error what lead to its removal
%         \end{itemize}
% \end{itemize}
%
%
% \section{Implementation}
% The basic idea is to write into the aux file and save a variable whenever a label is referenced, so that the information is available in the current and in the next run. The label information is passed from the label command to the newline command. The newline command uses the label and the reference information to possibly add a \cmd{\notag} command, deciding if it is referenced or not.
%
% ^^A Calling the old newline command leads to a redefinition (overwriting, not patching) of the (current) newline command (proved by using \cmd{\show}\cmd{\\} before and after the call to amsmath's newline command) when using local definitions. As amsmath behaves like a virus then, the patched newline function is only called for the first newline. After the first call, amsmath's hijacked newline is used directly without giving a chance to do something useful. To correct this behavior, the newline command would need to be patched again. Unfortunately, \cmd{\let} cannot be used to save the current command, because the copy generated by \cmd{\let} is deleted by amsmath (WTF?). To avoid all this, some global definitions are used.

%\begin{macro}{\csxdefaux}
% This command is similar to the \cmd{\csxdef} command from the package etoolbox, but instead of defining the command immediately, it is defined in the next run by writing it to the aux file. The name is given by the first argument (which may not have a leading backslash). The second argument is the replacement text. This command would be a candidate for inclusion into etoolbox. \cmd{\ifcsdef} can be used to check, if the command has been defined. See also http://tex.stackexchange.com/a/49035
% Usage: \cmd{\csxdefaux\{csname\}\{replacement\}}
%    \begin{macrocode}
\def\csxdefaux#1#2{%
%    \end{macrocode}
	% The expandafter commands are used to first expand the \cmd{\csname}-\cmd{\endcsname}. Then there is a command definition left, where the command and its replacement (which can also be a command) are protected by \cmd{\string} to create the command in the next run (when the aux file is read) and not in the current run (when the aux file is written).
%    \begin{macrocode}
	\protected@write\@mainaux{}{%
		\csxdef{\detokenize{#1}}{#2}%
	}%
}
%    \end{macrocode}
%\end{macro}

%\begin{macro}{\csxdefall}
% This command simply combines the commands \cmd{\csxdef} and \cmd{\csxdefaux}. This command would be a candidate for inclusion into etoolbox.
%    \begin{macrocode}
\def\csxdefall#1#2{%
	\csxdefaux{#1}{#2}%
	\csxdef{#1}{#2}%
}
%    \end{macrocode}
%\end{macro}

%\begin{macro}{\ifcsedef}
% This command is similar to the \cmd{\ifcsdef} test from the package etoolbox, but the command sequence gets fully expanded before it is evaluated.. This command would be a candidate for inclusion into etoolbox.
%    \begin{macrocode}
\def\ifcsedef#1#2#3{%
	\edef\autonum@ifcsedefTemp{#1}%
	\expandafter\ifcsdef\expandafter{\autonum@ifcsedefTemp}{#2}{#3}%
	\undef{\autonum@ifcsedefTemp}%
}
%    \end{macrocode}
%\end{macro}

% This is needed to not get overwritten by other packages. The package autonum only overwrites some commands whose name start with \cmd{\autonum}. Other commands are only patched, so the currently valid command gets called, too. So although not very polite, this behavior seems reasonable.
%    \begin{macrocode}
\AtBeginDocument{%
%    \end{macrocode}
	% Most of amsmath's environments are redefined. The environments aligned and gathered are not redefined, as it is unclear, how the numbering should work.
%    \begin{macrocode}
	%^^A\forcsvlist{\autonum@patchBlockEnvironment}{gathered,aligned}%
%^^A	\ifdef{\multlined}{%
%^^A		\autonum@patchBlockEnvironment{multlined}%
%^^A	}{}%
%    \end{macrocode}
	% If align is redefined before flalign or alignat, autonum.dtx does not build anymore. The error reason is unknown. As the error disappears when align is redefined after both, there is no motivation in finding the underlying problem.
%    \begin{macrocode}
	\newlength{\autonum@environmentWidth}%
	\forcsvlist{\autonum@patchParametrizedFullEnvironment}{alignat}%
	\forcsvlist{\autonum@patchFullEnvironment}{equation,gather,multline,flalign,align}%
%    \end{macrocode}
	%
	% Patch the environment delimited by \cmd{\[} and \cmd{\]}.
%    \begin{macrocode}
% 	\autonum@patchShortcutEnvironment
	\def\[#1\]{%
		\begin{equation}#1\end{equation}%
	}%
%    \end{macrocode}
	%
	% Support the normal \cmd{\ref} command and, if available, the \cmd{\cref} command from cleveref.
%    \begin{macrocode}
	\autonum@generatePatchedReference{ref}%
	\ifdef{\cref}{%
		\autonum@generatePatchedReferenceCSL{cref}%
	}{}%
%    \end{macrocode}
%    \begin{macrocode}
}
%    \end{macrocode}

%\begin{macro}{\autonum@patchEnvironment}
% Patch a mathematical environment to automatically show an equation's number, if a part is referenced and do not use a number otherwise. For completeness, the original definition (numbering every part of an displayed equation structure) is made available using a different name.  Do not redefine environment before getting the original label and newline commands. Use center as the default parameter, as a center environment is a neutral element regarding the subcommands' definitions.
%    \begin{macrocode}
% \def\autonum@patchEnvironment#1{
\def\autonum@patchFullEnvironment#1{%
	\autonum@saveEnvironmentSubcommands{#1}{center}{}%
	\autonum@patchEnvironmentHelper{#1}{0}%
}
\def\autonum@patchParametrizedFullEnvironment#1{%
	\autonum@saveEnvironmentSubcommands{#1}{center}{1}%
	\autonum@patchEnvironmentHelper{#1}{1}%
}
\def\autonum@patchBlockEnvironment#1{%
	\autonum@saveEnvironmentSubcommands{#1}{equation*}{}%
	\autonum@patchEnvironmentHelper{#1}{0}%
}
\def\autonum@patchEnvironmentHelper#1#2{%
	\autonum@renameEnvironment{#1}{#2}%
	\autonum@changeEnvironment{#1}{#2}%
	\autonum@generatePatchedLabel{#1}%
	\autonum@generatePatchedNewline{#1}%
}
%    \end{macrocode}
%\end{macro}

%\begin{macro}{\autonum@saveEnvironmentSubcommands}
% This code is to save the newline code used in a mathematic display environment so that it can be used later. This is necessary, as saving it in the instance of the environment, where it should be used, does not work. Allow for two arguments, to enable putting the alignment building blocks into an equation environment (see amsmath documentation). The third argument is needed for environments which have arguments themself.
%    \begin{macrocode}
\def\autonum@saveEnvironmentSubcommands#1#2#3{%
	\begin{textblock}{1}[1,1](0,0)%
		\begin{#2}%
			\begin{#1}#3%
			%    \end{macrocode}
				% Avoid underfull hbox warning in multline, by putting content of the correct size in it. The correct size can only be measured here, as the values might change due to the beginning of environments.
			%    \begin{macrocode}
				\deflength{\autonum@environmentWidth}{\linewidth-\multlinegap-\multlinegap}%
				\hspace{\autonum@environmentWidth}%
%    \end{macrocode}
				% Using global here is necessary to get the information out of the environment.
%    \begin{macrocode}
				\global\cslet{autonum@newline#1}\\%
%    \end{macrocode}
				% Use \cmd{\notag} to not increase the equation counter (otherwise the first equation shown would not have number 1).
%    \begin{macrocode}
				\notag%
%    \end{macrocode}
				% For multline, check, that autonum's label command is undefined, because this means, that the first pass (measuring pass) of the environment is active. This is to avoid getting the \cmd{\label} command of the second pass (displaying pass), where the \cmd{\label} command is set to the \href{http://tex.stackexchange.com/a/59088/7323}{null definition}. The multline environment seems to need the first pass, whereas other environments seem to need the second pass, so adapt to the environments.
%    \begin{macrocode}
				\ifboolexpr{not test {\ifstrequal{#1}{multline}} or test {\ifcsundef{autonum@label#1}}}{%
					\global\cslet{autonum@label#1}{\label}%
				}%
%    \end{macrocode}
				% Use \cmd{\notag} again, to avoid an error with the gathered environment. WTF?
%    \begin{macrocode}
				\notag%
			\end{#1}%
		\end{#2}%
	\end{textblock}%
}
%    \end{macrocode}
%\end{macro}

%\begin{macro}{\autonum@renameEnvironment}
% Rename the old environment to be accessible with an appended + by saving the original environment using a different name. The first argument contains the environment's name, the second argument contains the number of arguments the environment has.
%    \begin{macrocode}
\def\autonum@renameEnvironment#1#2{%
	\csletcs{autonum@#1Old}{#1}%
	\csletcs{autonum@end#1Old}{end#1}%
	\newenvironment{#1+}[#2]{%
		\csuse{autonum@#1Old}%
	}{%
		\csuse{autonum@end#1Old}%
	}%
}
%    \end{macrocode}
%\end{macro}

%\begin{macro}{\autonum@changeEnvironment}
% Now change the environment. This command only supports displayed equation structures and is not suited for other environments (as e.g. figures). The second argument contains the number of arguments the redefined environment has.
%    \begin{macrocode}
\def\autonum@changeEnvironment#1#2{%
%    \end{macrocode}
		% Although Amsmath's environment is executed twice (for measuring and for painting), the content here is executed only once. The following if is only needed to distinguish between environments without (e.g. equation) and with one parameter (e.g. alignat).
%    \begin{macrocode}
	\ifnum #2=0%
		\renewenvironment{#1}{%
			\autonum@startChangeEnvironment{#1}{}%
		}{%
			\autonum@endChangeEnvironment{#1}%
		}%
	\else
		\renewenvironment{#1}[1]{%
			\autonum@startChangeEnvironment{#1}{##1}%
		}{%
			\autonum@endChangeEnvironment{#1}%
		}%
	\fi
%    \end{macrocode}
		% Delete the starred versions of the environment, as they sometimes lead to strange errors a long time after using the starred version.  By deleting it, the error occurs at the right place.
%    \begin{macrocode}
	\global\csundef{#1*}%
	\global\csundef{end#1*}%
}
%    \end{macrocode}
%\end{macro}

%\begin{macro}{\autonum@startChangeEnvironment}
% Start the changed environment.
%    \begin{macrocode}
\def\autonum@startChangeEnvironment#1#2{%
%    \end{macrocode}
	% Prepare the label and the newline commands and begin the displayed equation environment.
%    \begin{macrocode}
	\autonum@saveSubcommands
	\csuse{autonum@#1Old}#2%
	\autonum@patchSubcommands{#1}%
}
%    \end{macrocode}
%\end{macro}

%\begin{macro}{\autonum@endChangeEnvironment}
% Close the changed environment.
%    \begin{macrocode}
\def\autonum@endChangeEnvironment#1{%
%    \end{macrocode}
	% Possibly hide the number of the last equation in the displayed equation environment, end the latter one and restore the subcommands.
%    \begin{macrocode}
	\autonum@possiblyHideNumber
	\csuse{autonum@end#1Old}%
	\autonum@restoreSubcommands
}
%    \end{macrocode}
%\end{macro}

%\begin{macro}{\autonum@saveSubcommands}
% Save the current newline and label commands.
%    \begin{macrocode}
\def\autonum@saveSubcommands{%
	\let\autonum@labelNormal\label%
	\let\autonum@newlineNormal\\%
}
%    \end{macrocode}
%\end{macro}

%\begin{macro}{\autonum@patchSubcommands}
% Patch the label command, as some special data has to be saved with each usage. In order to support multi-line equations, the counter must be increased in every line, as every line is a possible reference target. Therefore, \\ has to be overwritten, too. This must be global, as amsmath is very annoying with overwriting local definitions, e.g. in align environments.
%    \begin{macrocode}
\def\autonum@patchSubcommands#1{%
	\global\letcs{\label}{autonum@patched#1Label}%
%    \end{macrocode}
	% Do not patch the newline command in a multline environment, as only the last line may get a \cmd{\notag} command, because all lines basically build one equation (see also amsmath's documentation, section 3.3).
%    \begin{macrocode}
	\ifstrequal{#1}{multline}{%
	}{%
	 	\global\letcs{\\}{autonum@patched#1Newline}%
	}%
}
%    \end{macrocode}
%\end{macro}

%\begin{macro}{\autonum@restoreSubcommands}
% Restore the newline and label commands. This must be global, as it had been overwritten globally in \cmd{\autonum@patchSubcommands}.
%    \begin{macrocode}
\def\autonum@restoreSubcommands{%
	\global\let\label\autonum@labelNormal%
	\global\let\\\autonum@newlineNormal%
}
%    \end{macrocode}
%\end{macro}

%\begin{macro}{\autonum@generatePatchedLabel}
% Use an extra command to patch the used label command for efficiency.
%    \begin{macrocode}
\def\autonum@generatePatchedLabel#1{%
	\csdef{autonum@patched#1Label}##1{%
	%    \end{macrocode}
		% The labeling information is needed in the newline command. Therefore, the following variable is used to store it until the next newline command. As the definition is local and every line in an multi-line displayed math environment has its own group, the variable does not have to be deleted explicitly.
	%    \begin{macrocode}
		\ifdef{\autonum@currentLabel}{%
			\PackageError{autonum}{Two succeeding \string\label's detected}{Did you forget a \string\\?}%
		}{%
			\def\autonum@currentLabel{##1}%
		}%
%    \end{macrocode}
		% Only call the original label command if the label gets referenced. This obviously is identical if the reference is located before the label. It is also identical if the reference is located after the label, as the \cmd{\\} or \cmd{\endenvironment} commands which follow the \cmd{\label} would suppress the numbering anyway in the first pass. In the second pass, the information about referencing is the same as if only the content of the following if-command were available.
%    \begin{macrocode}
		\ifcsedef{autonum@##1Referenced}{%
%    \end{macrocode}
		% The environment's original label command is called to do the real labeling. As it checks for erroneous succeeding labels using \cmd{df@label}, this variable has to be emptied before every call.
%    \begin{macrocode}
			\let\df@label\@empty%
			\csuse{autonum@label#1}{##1}%
		}{}
	}%
}
%    \end{macrocode}
%\end{macro}

%\begin{macro}{\autonum@generatePatchedNewline}
% This command generates patched newline commands for displayed math environments, so that they can simply be activated when needed.
%    \begin{macrocode}
\def\autonum@generatePatchedNewline#1{%
	\csdef{autonum@patched#1Newline}{%
		\autonum@possiblyHideNumber
		\csuse{autonum@newline#1}%
	}
}
%    \end{macrocode}
%\end{macro}

%\begin{macro}{\autonum@possiblyHideNumber}
% Define this command, which can hide the current line's number if the label is not referenced.
%    \begin{macrocode}
\def\autonum@possiblyHideNumber{
	\ifdef{\autonum@currentLabel}{%
		\ifcsedef{autonum@\csuse{autonum@currentLabel}Referenced}{%
		}{%
			\notag%
		}%
%    \end{macrocode}
		% The current label does not have to be cleaned, as every line is a separate cell \href{http://tex.stackexchange.com/q/58190}{defining a local group} in an displayed math environment.
%    \begin{macrocode}
	}{%
		\notag%
	}%
}
%    \end{macrocode}
%\end{macro}

%\begin{macro}{\autonum@generatePatchedReference}
% This command can patch reference commands with a normal input argument.
%    \begin{macrocode}
\def\autonum@generatePatchedReference#1{%
	\autonum@generatePatchedReferenceGeneral{#1}{autonum@use}%
}
%    \end{macrocode}
%\end{macro}

%\begin{macro}{\autonum@generatePatchedReferenceCSL}
% This command can patch reference commands which expect a comma separated list as input argument.
%    \begin{macrocode}
\def\autonum@generatePatchedReferenceCSL#1{%
	\autonum@generatePatchedReferenceGeneral{#1}{forcsvlist}%
}
%    \end{macrocode}
%\end{macro}

%\begin{macro}{\autonum@generatePatchedReferenceGeneral}
% This command can patch arbitrary reference commands. The patch logic can be different to patching the label command, as the references have to be patched only once, so optimizing for speed is counter-productive.
%    \begin{macrocode}
\def\autonum@generatePatchedReferenceGeneral#1#2{%
	\csletcs{autonum@reference#1Old}{#1}%
	\csdef{#1}##1{%
		\csuse{#2}{\autonum@markLabelAsReferenced}{##1}%
%    \end{macrocode}
		% Call the old reference command.
%    \begin{macrocode}
		\csuse{autonum@reference#1Old}{##1}%
	}%
}
%    \end{macrocode}
%\end{macro}

%\begin{macro}{\autonum@markLabelAsReferenced}
% This is a simple helper macro to mark a label as referenced. The reference information is stored into a variable (for the current run) and into the aux file (for the next run), so it does not matter if the reference is used before or after the definition of the label. Saving into a variable saves one compilation run, although still up to three are needed to get everything right.
%    \begin{macrocode}
\def\autonum@markLabelAsReferenced#1{%
	\csxdefall{autonum@#1Referenced}{}%
}
%    \end{macrocode}
%\end{macro}

%\begin{macro}{\autonum@use}
% This is a simple helper macro which can be used like a function handle similar to csuse but expecting a macro instead of a macro's name.
%    \begin{macrocode}
\def\autonum@use#1#2{%
	#1{#2}%
}
%    \end{macrocode}
%\end{macro}

%\begin{macro}{\autonum@patchShortcutEnvironment}
% Use a counter to numerate all \cmd{\[}-\cmd{\]}-environments linearly.
%    \begin{macrocode}
\newcounter{autonum@counter}
%    \end{macrocode}
% \cmd{\[} and \cmd{\]} are redefined as the correct one of \texttt{equation} and \texttt{align}. Due to the improved numbering, the old environment's capabilities are basically a subset of the new capabilities.
%    \begin{macrocode}
\def\autonum@patchShortcutEnvironment{%
	\def\[##1\]{%
%    \end{macrocode}
		% This command checks if the current environment only consists of one line without counting lines in sub-environments. The default will result in an align environment, as incorrectly using an equation instead of an correct align results in a compile error.
%    \begin{macrocode}
		\ifcsedef{autonum@\Roman{autonum@counter}HasExactlyOneLine}{%
			\autonum@useWithMultipleLineDetection{equation}{##1}%
		}{%
			\autonum@useWithMultipleLineDetection{align}{##1}%
		}%
		\stepcounter{autonum@counter}%
	}%
}
%    \end{macrocode}
%\end{macro}

%\begin{macro}{\autonum@useWithMultipleLineDetection}
% This function uses an environment defined by the first argument to display the content given in the second argument. A multiple-line detection is activated, to set a variable if more than one line is used.
%    \begin{macrocode}
\def\autonum@useWithMultipleLineDetection#1#2{%
	\begin{#1}%
		\autonum@patchParentheses
%    \end{macrocode}
	% Use global as this is in the middle of the first local group of a math environment.
%    \begin{macrocode}
		\global\let\autonum@patchedNewline\\%
%    \end{macrocode}
	% Set the multipleLines variable if a newline is used. Do not use the newline for equations, as this results in "There's no line here to end" errors. This is ok, as if there is a newline in an environment, which is currently an equation, it is wrong anyway and should be set as an align environment. For that it is enough to set the multipleLines variable.
%    \begin{macrocode}
		\ifstrequal{#1}{align}{%
			\gdef\\{%
				\autonum@patchedNewline
				\gdef\autonum@multipleLines{}%
			}%
		}{%
			\gdef\\{%
				\gdef\autonum@multipleLines{}%
			}%
		}
%    \end{macrocode}
		% Set the environment's content and reset the newline command.
%    \begin{macrocode}
		#2%
		\global\let\\\autonum@patchedNewline%
%    \end{macrocode}
		% Store information if the current math environment. The roman number is used, as there might be no label and if there is one, it might not be available at the beginning of the environment. Delete the multipleLines variable, to avoid influencing the next \cmd{\[}-\cmd{\]}-environment, as the variable must be global.
%    \begin{macrocode}
		\ifdef{\autonum@multipleLines}{%
			\global\undef{\autonum@multipleLines}%
		}{%
			\csxdefaux{autonum@\Roman{autonum@counter}HasExactlyOneLine}{\Roman{autonum@counter}}%
		}%
		\autonum@restoreParentheses
	\end{#1}%
}
%    \end{macrocode}
%\end{macro}

%\begin{macro}{\autonum@patchParentheses}
% This function patches the left and the right parentheses.
%    \begin{macrocode}
\global\def\autonum@patchParentheses{%
	\autonum@patchParenthesis{(}{Left}{}%
	\autonum@patchParenthesis{)}{Right}{end}%
}
%    \end{macrocode}
%\end{macro}

%\begin{macro}{\autonum@patchParenthesis}
% This function patches a parenthesis given in the first argument with a name partly given in the second argument by using the third argument.
%    \begin{macrocode}
\global\def\autonum@patchParenthesis#1#2#3{%
	\ifcsdef{#1}{%
		\global\csletcs{autonum@old#2Parenthesis}{#1}%
	}{}%
	\global\csletcs{#1}{#3split}%
}
%    \end{macrocode}
%\end{macro}

%\begin{macro}{\autonum@restoreParentheses}
% This function restores the left and the right parentheses.
%    \begin{macrocode}
\global\def\autonum@restoreParentheses{%
	\autonum@restoreParenthesis{(}{Left}%
	\autonum@restoreParenthesis{)}{Right}%
}
%    \end{macrocode}
%\end{macro}

%\begin{macro}{\autonum@restoreParenthesis}
% This function restores a parenthesis given in the first argument with the name given in the second argument.
%    \begin{macrocode}
\global\def\autonum@restoreParenthesis#1#2{%
	\ifcsdef{autonum@old#2Parenthesis}{%
		\global\csletcs{#1}{autonum@old#2Parenthesis}%
		\global\csundef{autonum@old#2Parenthesis}%
	}{}%
}
%    \end{macrocode}
%\end{macro}

%\StopEventually{^^A
%  \PrintChanges
%^^A  \PrintIndex
%}
%
%\Finale